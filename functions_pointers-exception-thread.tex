\section{Puntatori e Funzioni}
Variabili che contengono indirizzi a funzioni. Questi servono quando il programma deve scegliere quale funzione chiamare fra diverse possibili, e la scelta non è definita a priori ma dipende dai dati del programma stesso.

Questo processo si chiama late binding: gli indirizzi delle funzioni da chiamare non vengono risolti al momento della compilazione, come avviene normalmente (early binding) ma al momento dell'esecuzione.

\begin{lstlisting}[language=c++]
    int a(double,int){...}
    int b(double,int){...}
    int f(int i,int j){return i*j;}
    int f2(int i, int j){return i+j;}
    double media_generale(int i, int j,int(*pf)(int,int) ){ 
        return double(pf(i,j))/2; 
    }
    main(){
        int (*pf)(int,int);
        int (*pfunz)(double, int);        
        ...
        if(i%2==0){
            pfunz=a;
        }else
            pfunz=b; 
        }   
        cout<<pfunz(2.0,3); //or cout<<(*pfunz)(2.0,3);

        pf=f;
        cout<<pf(2,3)<<endl; // or (*pf)(2,3);  //6
        cout<<media_generale(2,3,f)<<endl;      //3
        cout<<media_generale(2,3,f2)<<endl;     //2.5
    }
\end{lstlisting}

\section{Eccezioni}
Gli errori di runtime sono quegli errori che NON possono essere
rilevati in fase di compilazione, perché si manifestano solo durante
la fase di esecuzione del programma e solo in alcune particolari
circostanze, ossia ai veri casi di “eventi eccezionali”.
\paragraph{}
Alcuni esempi:

\begin{itemize}
    \item[--] divisione per 0;
    \item[--] pathname non valido o che fa riferimento ad una periferica che in un certo momento risulta scollegata;
    \item[--] overflow su un tipo di dato
    \item[--] operazione di casting non valido
\end{itemize}

Si tratta di errori pericolosi perché se non gestiti, possono generare
anomalie di funzionamento che possono determinare persino il
blocco inaspettato del programma.

Un’eccezione non necessariamente deve essere gestita all’interno
della stessa funzione in cui viene sollevata. Questo è reso possibile dal meccanismo proprio delle eccezioni, il
quale prevede che un’eccezione sollevata in una funzione e che non
venga in essa gestita, sia propagata all’indietro verso il chiamante,
si dice risalendo lo stack delle chiamate.


\subsection{throw}
Per gestire questi errori i linguaggi OOP implementano il meccanismo delle eccezioni, il quale consente di sollevare, \verb|throw|, un'eccezione quando si verifica un errore a runtime. Questo meccanismo si traduce nella creazione di un oggetto appartiene ad una classe particolare, dipendente dallo
specifico errore di runtime che si è verificato.

\subsection{try-cath}
Per "intercettare" l'errore in C++ viene fornito il costrutto \verb|try-cath| il quale permette di sospendere il programma nel punto dove si è verificato l'errore per saltare al pezzo di codice che gestisce l'errore.

\begin{lstlisting}[language=c++]
try{
    int x, y;
    //insert from keyboard
    if(y == 0) throw "Division by zero!";   //or division(x,y);
}catch(const char* msg){
    cerr<<msg<< endl;
}catch(){ }
...
\end{lstlisting}

\verb|try|:  blocco in cui si possono inserire le istruzioni che si vuole tenere “sotto controllo” perché durante l’esecuzione potrebbero generare degli errori a runtime.

\verb|catch|: blocco in grado di riconoscere e “catturare” un certo tipo di eccezione e che contiene le istruzioni per gestire l’errore di runtime che l’ha generata.
\paragraph{}
Quando nel blocco \verb|try| si genera un'eccezione si salta ai sui blocchi catch associati, partendo dal primo a scendere (dall'eccezione più specifica a quella più generale). Alla sua uscita il flusso del programma prosegue oltre l’ultimo blocco catch.


\subsection{stdexcept}
Questo header definisce un insieme di eccezioni standard che sia la libreria che i programmi possono utilizzare per segnalare errori comuni.
Includere nell'intestazione:\verb|#include <stdexcept>|.

\begin{itemize}
    \item Logic errors 
        \begin{itemize}
            \item \verb|logic_error|
            \item \verb|domain_error|
            \item \verb|invalid_argument|
            \item \verb|length_error|
            \item \verb|out_of_range|            
        \end{itemize}

    \item Runtime errors
        \begin{itemize}
            \item \verb|runtime_error|
            \item \verb|range_error|
            \item \verb|overflow_error|
            \item \verb|underflow_error|            
        \end{itemize}
\end{itemize}


\begin{lstlisting}[language=c++]
#include <stdexcept>
using namespace std;

int divideINT(int a, int b) {
    if (a < 0 || b < 0) {
        throw invalid_argument( "received negative value" );
    }
    return a/b;
}
void test(){
    int x=1, y=0; 
    try {
        cout << a << "/" << b << "=" << divideINT(a, b) << endl;
    } catch( const invalid_argument& e ) {
        cout << e.what(); << endl //or cout << "a or b negative" << endl;    //cout << e << endl; //bind ERROR    
    } cathc(...){   //otherwise
        cout << "other wrong error";
    }
}
int main(){ test(); }
\end{lstlisting}

\section{Multithreading in C++}
Dal C++ 11 si ha la casse \verb|thread|, la quale abilita alla scrittura di programmi che possono eseguire parti del codice in parallelo e non più in maniera sequenziale.

La classe \verb|thread| ha le seguenti funzioni:
\begin{itemize}
    \item detach
    \item get\_id
    \item join
    \item joinable
    \item join
    \item joinable
    \item native\_handle
    \item operator= (move semantics)
    \item hardware\_concurrency()
\end{itemize}


\begin{lstlisting}[language=c++]
#include <iostream>
#include <thread> 
using namespace std;

int a(){  /*do something...*/ }
int b(int count){ /*do something...*/ }

int main(){
    unsigned int c = thread::hardware_concurrency();
    thread t1 (a);
    thread t2;
    t2 = thread(b,0);

    t1.join();
    t2.join();
}
   
\end{lstlisting}

\subsection{Data race}
Si verifica quando due o più threads in un singolo processo accedono simultaneamente alla stessa area di memoria, almeno un accesso è eseguito in scrittura e non vi sono blocchi per l'accesso esclusivo. Questo significa che il valore da un momeno all'altro può essere cambiato ed alterare il funzionamento del programma.

Proprio per questo motivo in C++ vengono introdotti due blocchi: \verb|atomic| e \verb|mutex|.

\begin{itemize}
    \item \verb|atomic| rende atomiche le operazioni su una certa variabile;
    \item \verb|mutex| Definisco delle sezioni critiche come mutuamente esclusive nel codice;
\end{itemize}

\begin{lstlisting}[language=c++]
//first resolution with atomic 
#include <iostream>   
#include <thread>
#include <atomic>
using namespace std;

static std::atomic<int> count(0);
//static int count=0;
void inc_thread(){
    for(int i=0;i<10000;i++) count++;
}
void dec_thread(){
for (int i=0;i<10000;i++) count--;
}
main(){
    std::thread inc(inc_thread);
    std::thread dec(dec_thread);
    inc.join();
    dec.join();
    std::cout<<"counter"<<count;
    return 0;
} 
\end{lstlisting}

\begin{lstlisting}[language=c++]
//second resolution with mutex 
#include <iostream> 
#include <thread>
#include <mutex>
using namespace std;

mutex myMutex;
static int condivisa=0;
void inc_thread(){
    for(int i=0;i<10000;i++){
    myMutex.lock(); condivisa++; myMutex.unlock();}
}
void dec_thread(){
    for (int i=0;i<10000;i++){
    myMutex.lock(); condivisa--; myMutex.unlock();}
}
int main (){
    std::thread inc(inc_thread);
    std::thread dec(dec_thread);
    inc.join();
    dec.join();
}
\end{lstlisting}


\subsection{Problemi della concorrenza: deadlock e starvation}

\paragraph{Deadlock}
Situazioni in cui uno o piu’ processi/thread si bloccano:
\begin{itemize}
    \item Omettere l’unlock un mutex
    \item Terminazione inattesa di una funzione (lancio di eccezioni), soluzione: \verb|lock_guard<mutex>|
    \item Funzioni annidate che chiamano lo stesso mutex, soluzione: \verb|recursive_mutex|
    \item Ordine di locking di diversi mutex , soluzione: \verb|ock(mutex, mutex)|
\end{itemize}

\paragraph{Starvation} 
La politica di accesso impedisce ad un processo di accedere alla risorsa

\paragraph{}
Esistono nella letteratura altre primitive usando per
gestire la concorrenza come semafori e monitor ma non sono forniti dallo standard di C++ che
preferisce fornire primitive di livello più basso.