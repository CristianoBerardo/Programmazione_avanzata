\section{Standard Template Library}

La Standard Template Library (STL) è una libreria software per il linguaggio di programmazione C++ che definisce quattro componenti principali: contenitori, iteratori, algoritmi e funtori.
\paragraph{I contenitori } si dividono in sequenziali e associativi:

\begin{itemize}
    \item[--] I contenitori standard sequenziali includono \verb|vector|, \verb|list|e \verb|deque|;
    \item[--] I contenitori standard associativi sono invece: \verb|set|, \verb|multiset|,  \verb|map |e \verb|multimap |.
\end{itemize}

\begin{table}[h]
    \centering
    \begin{tabular}{c|c}
    \hline
            Contenitore&Descrizione\\
    \hline
         \verb|vector|  &  Array dinamico capace di ridimensionarsi, elementi contigui\\
                        & accesso, inserimento e cancellazione in coda \textbf{O(1)}\\
                        & inserimento in testa, ricerca, cancellazione \textbf{O(n)} \\
    \hline
         \verb|list|    & lista bidirezionale, elementi non contigui in memoria \\                         
                        & accesso e ricerca mediante \textit{iteratore} \textbf{O(n)} \\
                        & inserimento e cancellazione  contigui\\
    \hline\hline
         \verb|set| & insieme ordinato (piccolo-grande) senza duplicati, elementi non contigui\\
                    & il valore degli elementi non può essere modificato\\
                    & ricerca, inserimento e cancellazione \textbf{O(log(n))}\\
                    &implementazione BST\\
                    &\verb|unordered_set| nessun ordinamento, no duplicati in media \textbf{O(log(1))}\\
                    &implementazione hash table\\
                    \hline
         \verb|multiset|& uguale al set ma consente duplicati\\
                        &\verb|unordered_multiset| uguale al unordered\_set ma con duplicati \\
                    \hline
         \verb|map| & array associativo <key, value> ordinato rispetto alla chiave \\
                    & generalmente \textbf{O(log(n))}\\
                    &\textbf{O(log(1))} se si accede tramite operator[]\\
                    &\verb|unordered_map| non ordinato, \textbf{O(log(1))} \\
                    \hline
         \verb|multimap|& come map, ma può contenere più valori uguali \\
                
    \end{tabular}
    \caption{Contenitori STL}
    \label{tab:my_label}
\end{table}


\section{Iteratori}
Oggetto che consente di visitare tutti gli elementi contenuti in un altro oggetto, tipicamente un contenitore, senza doversi preoccupare dei dettagli di una specifica implementazione. Un iteratore è talvolta chiamato cursore, specialmente nel contesto delle basi dati.

\begin{lstlisting}[language=c++]
#include <iostream>
using namespace std;
#include<list>
int main(int argc, char *argv[]){
    list<int> l;
    l.push_back(2);
    l.push_front(3);
    list<int>::iterator it; //c++98
    for(it=l.begin();it!=l.end();++it)
        cout<<*it;
    for(auto a=l.begin();a!=l.end();++a) //c++11 e successivi
        cout<<*a;
    for(auto& e:l)
        e++;                //out: 43
    cout<<l.front() << l.back();    //out: 43
    l.push_back(2);
    l.push_front(5);
    l.pop_back();
    l.pop_front(); 

    auto a=l.cbegin();//const_iterator  l=74
    l.push_front(8);l.push_front(9);    //l=9874
    cout<<endl<<*a;     //7

    list<int> l2;
    l2.splice(l2.cbegin(),l,a,l.cend());    //l: 98 and l2: 74
    l2.swap(l);                             //l: 74 and l2: 98

    cout<<l.size();     //2
    
    l.reverse();    //47
    l.sort();       //47 (773344->334477)
    l.unique();     //47 (347)
    l2.sort()       //89
    l.merge(l2);    //l: 4789 l2: 

    //--reverse iterator--

    list<int>::reverse_iterator rit;
    for(rit=l.rbegin();rit!=l.rend();++rit)
        cout<<*rit;                            //9874
    for(auto a=l.rbegin();a!=l.rend();++a)
        cout<<*a;

    l.resize(3);            //l: 478
    l.remove(7);            //l: 48
    l.erase(l.begin());     //l: 8
    a=l.emplace(l.cbegin(),11); //l: 118
    l.push_front(13)            //l: 13118

    l2.assign(a,l.cend());      //l: 13118 l: 118

    list<int> l3(l);        //l3: 13118
    l=l2;                   //l: 118    l2:118

    l.clear();              //l: 
    cout<<endl<<l.max_size();   //768614336404564650
}
    
\end{lstlisting}


\section{set \& map}
Sono ordinati e usano operator<, non hanno il concetto di front e back, ma si usa \verb|insert|. Per utilizzare l'inserimento in map si deve utilizzare un nuovo operatore \verb|pair<T,T>|

\begin{lstlisting}[language=c++]
#include <iostream>
using namespace std;
#include<set>
#include<map>

int main(int argc, char *argv[]){
    set<A> s;
    s.insert(A()); // funziona solo se definito A::operator<
    map<A,int> m;
    m.insert(pair<A,int> (A(),2));// funziona solo se definito A::operator<
    map<int,A> ma;
    ma.insert(pair<int,A> (3,A()));//funziona,operator< definito per int
    ma[3]=A(); //equivalentemente per il map e' ridefinito operator[]            
}    
\end{lstlisting}


\section{multiset \& multimap}
Sono simili ai loro genitori: permettono di avere più elementi con lo stesso valore. Non può essere usato \verb|operator[]|.