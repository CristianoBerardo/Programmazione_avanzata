\section{Programmazione generica}
Si intende la possibilita’ data da un linguaggio di
rappresentare tipi e implementare algoritmi che
abbiano un tipo come parametro.
Il tipo parametrico viene poi specificato:
\begin{itemize}
    \item Tempo di compilazione (es. templates in C++)
    \item Tempo di esecuzione (es generics in Java)
\end{itemize}

Lo scopo è implementare algoritmi che siano
indipendenti dal tipo su cui operano.Idealmente al massimo livello di astrazione possibile.

\begin{lstlisting}[language=c++]
#include<string>
#include<utility>
#include<iostream>
using namespace std;

template <class T>
void my_swap(T& f, T& s) {
    T tmp = f;
    f = s;
    s = tmp;
}
int main(){
    int a = 3; int b = 4;
    my_swap<int> (a,b);     //a=4   b=3
    string s1 = "hello";
    string s2 = "world";
    my_swap<string> (s1,s2);  //s1=world   s2=hello
return 0;
}
\end{lstlisting}

\subsection{Utilizzo metodo clone:}
\begin{lstlisting}[language=c++]
class A{
    public:
    virtual A& operator=(const A& _a)=0;
    virtual A* clone()const=0;
    virtual ~A(){};
};
----
#include "A.h"
class B:public A{
    int i;
    public:
    B(int _i){i=_i;};
    B& operator=(const A& _b);
    B& operator=(const B& _b);
    B* clone()const;
    friend ostream& operator<<(ostream& os,const B& _b);
};
ostream& operator<<(ostream& os,const B& _b); 
----
#include "B.h"
B& B::operator=(const A& _b){
    A* temp=_b.clone();
    B b(*((B*)temp));
    *this=b;
    return *this;
}
B& B::operator=(const B& _b){
    i=_b.i;
    return *this;
}
B* B::clone()const{ return new B(*this);};
ostream& operator<<(ostream& os,const B& _b){return os<<_b.i;}

----

#include<string>
#include<utility>
#include<iostream>
using namespace std;
#include "B.h"

void my_swap (A &f, A &s ) {
    A* temp=f.clone();
    f=s;
    s=*temp;
    delete temp;
}
int main() {
    B x(33);
    B y(44);
    cout << x << y << endl;
    my_swap(x,y);
    cout << x << y << endl;
    return 0;
}
\end{lstlisting}