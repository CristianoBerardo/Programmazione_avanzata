\section{Bitset}
Template class della standard template library realizza un'array di bit, e l’implementazione permette di rappresentare i bit come bit fisici. 
Alcune funzioni della libreria algorithm, per l’elenco completo si veda il seguente \href{https://cplusplus.com/reference/bitset/bitset/}{link}

\begin{itemize}
    \item \verb|operator[]|: Access bit
    \item \verb|count|: Count bits set
    \item \verb|size|:	Return size 
    \item \verb|test|:	Return bit value
    \item \verb|any|:	Test if any bit is set
    \item \verb|none|:	Test if no bit is set
    \item \verb|all|:	Test if all bits are set
    \item \verb|set|:	Set bits
    \item \verb|reset|:	Reset bits
    \item \verb|flip|:	Flip bits
    \item \verb|to_string|:	Convert to string 
    \item \verb|to_ulong|:	Convert to unsigned long integer
    \item \verb|to_ullong|:	Convert to unsigned long long
\end{itemize}

\begin{lstlisting}[language=C++]
#include <iostream>
#include <bitset> 
#include <string>
#include <cstddef>        // std::size_t
using namespace std;
int main (){
    bitset<4> foo;    
    cout << foo.set() << '\n';       // 1111
    cout << foo.set(2,0) << '\n';    // 1011
    cout << foo.set(2) << '\n';      // 1111
    cout << foo << " as an integer is: " << foo.to_ulong() << '\n';     
    //1111 as an integer is: 15
    
    bitset<4> fo (string("1011"));
    cout << fo.reset(1) << '\n';    // 1001
    cout << fo.reset() << '\n';     // 0000

    bitset<4> f (string("0001"));
    cout << f.flip(2) << '\n';     // 0101
    cout << f.flip() << '\n';      // 1010

    bitset<4> a;
    a[1]=1;             // 0010
    
    cout << boolalpha;
    for (size_t i=0; i<a.size(); ++i)   //fase true false false
        cout << a.test(i) << ' ';
        
    a[2]=a[1];        // 0110

    cout << "Please, enter a binary number: ";
    cin >> num;
    if (foo.any())
        cout << foo << " has " << count() << " bits set.\n";    
        //0000000000010110
    else
        cout << foo << " has no bits set.\n";
    
    return 0;
}

\end{lstlisting}

Altri esempio con l'operatore shift \verb|<<|:

\begin{lstlisting}[language=C++]
#include <iostream>
#include <bitset> 

using namespace std;
int main (){
    bitset<64> b;
    cout<<b.size()<<endl;
    cout<<endl<<b<<" b";
    string s="101010101010111010101";
    cout<<endl<<s<<" s";
    b=bitset<64>(s); cout<<endl<<b<<" b=bitset<64>(s)";
    b<<=4; cout<<endl<<b<<" b<<=4";
    cout<<endl<<(b<<3)<<" b<<3";
    
    cout<<endl<<(b&(b<<3))<<" b&(b<<3) ";
    cout<<endl<<(b|(b<<3))<<" b|(b<<3)";
    //istringstream is2(s);
    //is2>>b; cout<<endl<<b<<" b";
    
    istringstream(s)>>b; cout<<endl<<b<<" s>>b";
    cout<<endl<<~b<<" ~b";
    cout<<endl<<(~b<<15)<<" ~b<<15)";
    b&=(~b<<15);cout<<endl<<b<<" b&=(~b<<15)";
    
    cout<<endl<<(b<<32)<<" b<<32";
    cout<<sizeof(long long)<<" ";
    cout<<(b<<32).to_ullong();
    cout<<endl<<b<<" b"<<b.to_ulong();
    //cout<<(b<<32).to_ulong();
    
    b.flip(); cout<<endl<<b<<" b.flip()";
    b.reset(42); cout<<endl<<b<<" b.reset(42)";
    if (!b[42]) b[41]=0; cout<<endl<<b<<" (!b[42]) b[41]=0"; 
    
    return 0;
}
\end{lstlisting}

\paragraph{Output:}

\begin{verbatim}
64
0000000000000000000000000000000000000000000000000000000000000000 b
101010101010111010101 s
0000000000000000000000000000000000000000000101010101010111010101 b=bitset<64>(s)
0000000000000000000000000000000000000001010101010101110101010000 b<<=4
0000000000000000000000000000000000001010101010101110101010000000 b<<3
0000000000000000000000000000000000000000000000000100100000000000 b&(b<<3)
0000000000000000000000000000000000001011111111111111111111010000 b|(b<<3)
0000000000000000000000000000000000000000000101010101010111010101 s>>b
1111111111111111111111111111111111111111111010101010101000101010 ~b
1111111111111111111111111111010101010101000101010000000000000000 ~b<<15)
0000000000000000000000000000000000000000000101010000000000000000 b&=(~b<<15)
0000000000010101000000000000000000000000000000000000000000000000 b<<328 5910974510923776
0000000000000000000000000000000000000000000101010000000000000000 b1376256
1111111111111111111111111111111111111111111010101111111111111111 b.flip()
1111111111111111111110111111111111111111111010101111111111111111 b.reset(42)
1111111111111111111110011111111111111111111010101111111111111111 (!b[42])b[41]=0

\end{verbatim}

\section{Espressioni lambda}
Le espressioni lambda si possono pensare come funzioni senza nome, esse sono definite la dove vengono anche usate e il loro punto essenziale è la cattura di variabili dal contesto in cui appaiono.
Queste espressioni lambda vengono spesso utilizzate come argomento
per le funzioni che accettano un oggetto chiamabile.

\paragraph{}
 Ciò può essere più semplice della creazione di una funzione con nome, che
verrebbe utilizzata solo quando passata come argomento. In tali
casi, le espressioni lambda sono generalmente preferite perché
consentono di definire gli oggetti funzione in linea.
\paragraph{}
In pratica una espressione lambda è un altro modo di definire in
C++ un entità invocabile che può essere usata in maniera diretta o
passata come argomento ad un’altra funzione, ma la cui
applicazione è di scarsa utilità al di fuori del contesto di definizione.
• In questi casi, la definizione di una espressione lambda risulta meno
prolissa della definizione di una funzione o un funtore ad hoc ed ha il
grande vantaggio di essere localizzata nel contesto d’uso.

\begin{verbatim}
    [ captures ] ( <params> ) <mutable> < -> > <return_type> { <body_of_functio> }
\end{verbatim}

La cattura specifica il modo di cattura per riferimento $[\&]$ o per valore/copia $[=]$. 
Nella clausola di cattura si può definire la cattura di default inserendo come primo paramentro $[\&]$ oppure $[=]$, facendo così però non si può più inserire $[\&identifier]$ oppure $[=identifier]$, perchè questi sono già stati catturati con il metodo di defalut.
L’uso del qualificare \verb|mutable| in questo contesto rende infatti possibile alterare il valore delle variabili catturate per valore nel corpo di istruzioni di una espressione lambda.In pratica, le alterazioni prodotte dal corpo della espressione lambda
persistono tra un’esecuzione e l’altra, ma non hanno effetti nel
contesto esterno.

\begin{lstlisting}[language=c++]
struct S { void f(int i); };

void S::f(int i) {
    [&, i]{};      // OK
    [&i]{};        // OK
    [&, &i]{};     // ERROR: i preceded by & when & is the default
    [=, this]{};   // ERROR: this when = is the default
    [=, *this]{ }; // OK: captures this by value. See below.
    [i, i]{};      // ERROR: i repeated
}
    
\end{lstlisting}

La cattura per valore, come discusso in precedenza, consente di
catturare per copia una variabile dal contesto esterno alla
espressione. Per default, tale copia è accessibile all’interno del corpo
di istruzioni in sola lettura, ma non è modificabile.
• Ciò consente di invocare la medesima espressione lambda più volte,
senza il rischio di alterare i valori catturati in origine dal contesto
esterno tra un esecuzione e l’altra.
• La cattura per riferimento, invece, consente di alterare liberamente
le variabili catturate dall’esterno. Come conseguenza di ciò, a
seguito dell’esecuzione dell’espressione lambda, il contesto di
invocazione risulta alterato. 

\begin{lstlisting}[language=c++]
#include <iostream>
#include <vector>
#include <numeric>      //for iota
#include <list>
#include <algorithm>
using namespace std;

class Amico{
    private:
    string nome;
    int eta;
    public:
    //operator ++ e -
};

void incEta(list<vector<Amico>>& gruppi){
    auto modify = [&gruppi](){
        for (auto& gr : gruppi) {
            for (auto& am : gr) ++am;
        }
    };
    modify();
}

int main(){
    auto somma = [](int x, int y) { return x + y; };
    //cout << "somma=" << somma << endl; //errore
    cout << "somma=" << somma(5, 8) << endl; 
    
    vector<int> vec(10);
    iota(v.begin(), v.end(),0);
    for(auto item: vec){ cout << "["<<item<<"]"; }
    cout << endl;
    // Trova un numero inferiore a un dato input
    int threshold = 10;
    auto it = find_if( vec.begin(), vec.end(),
    [threshold](int value) { return value < threshold; } );


    int num = 3;
    // trasforma il contenuto di vec moltiplicandolo per num
    transform(vec.begin(), vec.end(), vec.begin(),
                            [num] (const int& e) {
                            return e * num;
                            } );

    // rimuove tutti gli elementi minori di soglia
    int soglia = 10;
    //[soglia](int n) { return n < soglia; }
    vec.erase( remove_if(vec.begin(), vec.end(),
                                [soglia](int n) {
                                return n < soglia; 
                                } ), 
                                vec.end() );




    list <int> mylist{ 1, -2, 3, 4, -5, 8, 6};
    //VERSONE 1:: rimuove tutti gli elementi pari con una
    espressione lambda
    auto fremove = [&mylist] () {
    for(auto it = mylist.begin(); it != mylist.end(); ){
        if((*it)%2==0){
            auto it2remove = it;
            it++;
            mylist.erase(it2remove);
        } else {
            it++;
        }
    };
    fremove();



    //metodo remove_if delle liste
    void rimuoviPari1 (list<int>& l) {
        l.remove_if( [](int i) {return i%2==0;} );
    }
    //funzione remove_if di algorithm
    void rimuoviPari2 (list<int>& l) {
        remove_if(l.begin(), l.end(),
        [](int i) ->bool { return i%2==0; } );
    }
    rimuoviPari1(mylist);
    //rimuoviPari2(mylist);


    //VERSONE 3:: rimuove tutti gli elementi pari con una    espressione lambda
    auto rimuovi=[&mylist]()mutable{mylist.remove_if (pari); };
    //forma alternativa
    //auto rimuovi = [&]()mutable{mylist.remove_if (pari); };
    rimuovi();
    
    

    return 0;
}
    

    
\end{lstlisting}
