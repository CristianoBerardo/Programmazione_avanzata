\section{Ereditarieta’}
In C++ vi sono tre tipi di ereditarieta’:

\begin{itemize}
    \item public
    \item private
    \item protected
\end{itemize}
Queste sono anche le stesse keywords della visibilita’ nella classe.

\begin{table}[ht]
\caption{Ereditarietà}
\begin{center}
\begin{tabular}{c|c|c}
    Tipo di Ereditarietà &Casse base& Casse derivata\\
    \hline
    \multirow{3}{*}{Public}     &Public&Public\\
                                &Protected&Protected\\
                                &Private&Inacc.\\
    \hline

    \multirow{3}{*}{Protected}     &Public&Protected\\
                                &Protected&Protected\\
                                &Private&Inacc.\\
    \hline
    \multirow{3}{*}{Private}     &Public&Private\\
                                &Protected&Private\\
                                &Private&Inacc.\\
    \hline
\end{tabular}
\end{center}
\label{tab:multicol}
\end{table}

\subsection{Keyword virtual}
I metodi possono essere definiti virtual, in questo modo C++ riesce a dichiara il late binding (risolto run-time) rispetto al early binding (risolto a compilation-time).
\paragraph{}
Una classe è detta \verb|Puramente virtuale| se ha almeno un metodo virtual, ovvero la sua dichiarazione nella classe è seguita da "=0":

\begin{lstlisting}[language=c++]
class A{ 
    virtual int metodo()=0;
    ...
};
\end{lstlisting}

Una classe puramente virtuale NON si può istanziare.