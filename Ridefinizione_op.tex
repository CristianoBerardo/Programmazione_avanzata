\section{Ridefinizione degli operatori}
Il linguaggio C++ supporta la ridefinizione degli operatori. Gli operatori sono le operazioni che appaiono nelle espressioni.

Supponiamo di avere A a1,a2, a3, a4; nell’espressione
a1=(a2+a3)*(++a4); compaiono quattro operatori operator=,
operator+, operator* e operator++.
operator++ ha un solo operando mentre operator=, operator+,
operator* ne hanno due. 

\paragraph{}
Gran parte degli operatori si possono ridefinire o come metodi
della classe o come funzioni esterne, ma NON le due cose
assieme.
Quando si ridefiniscono come metodi il primo operando
corrisponde all’istanza chiamante e gli altri (se ve ne sono) a
parametri del metodo.
Quando si ridefiniscono come funzioni esterne alla classe TUTTI
gli operandi corrispondono a parametri della funzione. 
\paragraph{}
operator= si puo’ ridefinire SOLO come metodo ovvero

\begin{lstlisting}[language=c++]
    A& A::operator=(const A&)
\end{lstlisting}


\subsection{Ridefinizione operator<<}

\begin{lstlisting}[language=c++]
    class A{
        ...

        friend ostream& operator<<(ostream &os, const A& _a);
    }
    ostream& operator<<(ostream &os, const A& _a);
\end{lstlisting}

\subsection{Ridefinizione operator+}
\begin{lstlisting}[language=c++]
    class A{
        ...
        A operator+(const A&); 

        //modo da preferire
        friend A operator+(const A&,const A&);
    }
    A operator+(const A& _x, const A& _y);
\end{lstlisting}
\subsection{Ridefinizione operator-}

\begin{lstlisting}[language=c++]
    class A{
        ...
        friend A operator-(const A& _c1, const A& _c2); 
    }
    A operator-(const A& _c1, const A& _c2); 
\end{lstlisting}
\subsection{Ridefinizione operator*}
\begin{lstlisting}[language=c++]
    class A{
        ...
        friend A operator*(const A& _c1, const A& _c2);
    }
    A operator*(const A& _c1, const A& _c2); 
\end{lstlisting}
\subsection{Ridefinizione operator++}

\begin{lstlisting}[language=c++]
    class A{
        ...
        A& A::operator++();     //prefisso
        A A::operator++(int);   //postfisso
    }
\end{lstlisting}

\subsection{Distinzione sul tipo di ritorno}
Ci sono operatori che ritornano il risultato per valore e
operatori che ritornano il risultato per referenza. Come nel caso degli incrementi o decrementi prefissi o postfissi la distinzione si ha quando il risultato è utilizzato successivamente nella stessa espressione. 
\paragraph{}
Gli operatori che modificano l’oggetto chiamante
ritornando una reference. Esempio: operator+= e
operator+ il primo ritornerà A\& mentre il secondo A.


\subsection{Parola chiave friend}
Quando dentro una classe compare questa parola chiave la
riga NON e’ una dichiarazione di funzione o metodo ma piu’
semplicemente permette l’accesso alle parti private da parte di
una funzione o anche classe definita altrove. 
