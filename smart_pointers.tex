\section{Smart pointers}
Puntatori intelligenti che provvedono in automatico alla cancellazione della memoria. Non fanno parte delle caratteristiche basi del C++ e proprio per questo vengono forniti attraverso le librerie tra cui quella standard: \verb|#include <iostream>|.

Dopo essere stato inizializzato, il puntatore intelligente detiene la proprietà del puntatore non elaborato. Ciò significa che il puntatore intelligente è responsabile dell'eliminazione della memoria specificata dal puntatore non elaborato. Il distruttore del puntatore intelligente contiene la chiamata per l'eliminazione e poiché il puntatore intelligente viene dichiarato nello stack, il relativo distruttore viene richiamato quando il puntatore intelligente esce dall'ambito, anche se viene generata un'eccezione in un punto precedente dello stack.

Accedere al puntatore incapsulato utilizzando gli operatori del puntatore noti, \verb|->| e \verb|*|, di cui verrà eseguito l'overload per restituire il puntatore non elaborato incapsulato.

\paragraph{}
Nel linguaggio C++ moderno, i puntatori non elaborati vengono utilizzati esclusivamente in blocchi di codice piccoli con ambito limitato, nei cicli o nelle funzioni di supporto per le quali le prestazioni sono importanti e non può crearsi confusione circa la proprietà.

La differenza, rispetto ad altri linguaggi, consiste nel fatto che nessun Garbage Collector separato viene eseguito in background. La memoria viene gestita tramite le regole di ambito C++ standard in modo che l'ambiente di runtime risulti più veloce e più efficiente.
\paragraph{}
IMPORTANTE: Creare sempre puntatori intelligenti su una riga di codice separata e mai in un elenco di parametri, in modo da evitare anche le più piccole perdite di risorse dovute a alcune regole di allocazione dell'elenco di parametri.






\begin{lstlisting}[language=C++]
#include <iostream>
#include <memory> 

void UseRawPointer(){
    // Using a raw pointer -- not recommended.
    Song* pSong = new Song(L"Nothing on You", L"Bruno Mars"); 

    // Use pSong...

    // Don't forget to delete!
    delete pSong;   
}

void UseSmartPointer(){
    // Declare a smart pointer on stack and pass it the raw pointer.
    unique_ptr<Song> song2(new Song(L"Nothing on You", L"Bruno Mars"));
    
    unique_ptr<int> p;
    p.reset(new int(54));
    
    // Use song2...
    wstring s = song2->duration_;
    //...

} // song2 is deleted automatically here.

\end{lstlisting}


\subsection{\text{unique\_ptr}}
Classe template che permette di gestire una risorsa nella memoria dinamica garantendone l'ownership.
Può essere spostato a un nuovo proprietario, ma non copiato o condiviso. Sostituisce \verb|auto_ptr|, che è deprecato. 

\subsection{\text{shared\_ptr}}
Puntatore intelligente con conteggio dei riferimenti, per accedere al conteggio viene fornito il metodo \verb|use_count()|. Possiede più ownership, utilizzato ad esempio quando si restituisce una copia di un puntatore da un contenitore, ma si desidera conservare l'originale.

Il puntatore non elaborato non viene eliminato finché tutti i proprietari di \verb|shared_ptr| non sono usciti dall'ambito o non hanno ceduto in altro modo la proprietà.

\begin{lstlisting}[language=C++]
class A{
    shared_ptr<int> p;
    int owner(){
        return p.use_count();
    }
}; 
\end{lstlisting}

\subsection{\text{weak\_ptr}}
Puntatore intelligente per casi speciali da utilizzare insieme a \verb|shared_ptr|.
\verb|weak_ptr| fornisce l'accesso a un oggetto di proprietà di una o più istanze di \verb|shared_ptr|, ma non partecipa al conteggio dei riferimenti. Utilizzarlo quando si desidera osservare un oggetto, ma non è necessario che rimanga attivo.

Per accedere al contenuto e’ necessario copiarlo in
uno \verb|shared_ptr| usando un metodo lock().

\begin{lstlisting}[language=C++]
class A{
    weak_ptr<int> wp; 
    
    int get_wp(){
        shared_ptr<int> sp=wp.lock();
        return *sp;
    }
}; 
\end{lstlisting}


\subsection{Metodi utili}

\begin{lstlisting}[language=C++]
#include <iostream>
#include <memory> 
using namespace std;

class A{
    unique_ptr<int> wp; 
    unique_ptr<Data> wp_data; 
    
    p.reset(new int(54)); 
    cout << *p;

    p.reset()   // Free the memory before we exit function block.
    
    ---
    
    wp_data -> doSomething();   //Call a method on the object
    Data* data = wp_data.get();   //Pass raw pointer get()

    ---

    shared_ptr<int> p;
    int numer_owners =  p.use_count();
}; 
\end{lstlisting}